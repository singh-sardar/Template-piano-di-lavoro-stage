%----------------------------------------------------------------------------------------
%	STAGE DESCRIPTION
%----------------------------------------------------------------------------------------
\section*{Scopo dello stage}
% Personalizzare inserendo lo scopo dello stage, cioè una breve descrizione
Lo scopo di questo stage è quello di comprendere ed analizzare il funzionamento del protocollo ipermediale distribuito IPFS, nella sua versione pubblica e privata, individuandone le modalità di installazione, configurazione e gestione ed infine realizzare una rete sulla quale poter eseguire dei test. A completamento dello stage verrà creato un client che permetta, una volta loggato, di inserire  e gestire i file sulla rete. 

Lo studente avrà il compito di:
\begin{itemize}
\item Comprendere il funzionamento di IPFS;
\item Analizzare pro e contro della versione privata di IPFS (supporto, maturità, ecc);
\item Implementare il caso d’uso proposto;
\item Documentare le modalità di integrazione, i passi per l’installazione e i metodi fruibili con IPFS;
\item Realizzare una UI che permetta, una volta loggato, di inserire  e gestire i file sulla rete.  

\end{itemize}


