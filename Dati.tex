%----------------------------------------------------------------------------------------
%   USEFUL COMMANDS
%----------------------------------------------------------------------------------------

\newcommand{\dipartimento}{Dipartimento di Matematica ``Tullio Levi-Civita''}

%----------------------------------------------------------------------------------------
% 	USER DATA
%----------------------------------------------------------------------------------------

% Data di approvazione del piano da parte del tutor interno; nel formato GG Mese AAAA
% compilare inserendo al posto di GG 2 cifre per il giorno, e al posto di 
% AAAA 4 cifre per l'anno
\newcommand{\dataApprovazione}{Data}

% Dati dello Studente
\newcommand{\nomeStudente}{Harwinder}
\newcommand{\cognomeStudente}{Singh}
\newcommand{\matricolaStudente}{1143265}
\newcommand{\emailStudente}{harwinder.singh@studenti.unipd.it}
\newcommand{\telStudente}{+ 39 351 130 3604}

% Dati del Tutor Aziendale
\newcommand{\nomeTutorAziendale}{Fabio}
\newcommand{\cognomeTutorAziendale}{Canevarolo}
\newcommand{\emailTutorAziendale}{fabio.canevarolo@ifin.it}
\newcommand{\telTutorAziendale}{+ 39 342 764 9988}
\newcommand{\ruoloTutorAziendale}{}

% Dati dell'Azienda
\newcommand{\ragioneSocAzienda}{IFIN SISTEMI S.R.L.}
\newcommand{\indirizzoAzienda}{Via G. Medici 2/A, Padova}
\newcommand{\sitoAzienda}{https://ifin.it/}
\newcommand{\emailAzienda}{}
\newcommand{\partitaIVAAzienda}{P.IVA 01071920282}

% Dati del Tutor Interno (Docente)
\newcommand{\titoloTutorInterno}{Prof.}
\newcommand{\nomeTutorInterno}{Mauro}
\newcommand{\cognomeTutorInterno}{Conti}

\newcommand{\prospettoSettimanale}{
     % Personalizzare indicando in lista, i vari task settimana per settimana
     % sostituire a XX il totale ore della settimana
    \begin{itemize}
        \item \textbf{Prima Settimana (XX ore)}
        \begin{itemize}
            \item Incontro con persone coinvolte nel progetto per discutere i requisiti e le richieste
            relativamente al sistema da sviluppare;
            \item Verifica credenziali e strumenti di lavoro assegnati;
            \item Presa visione dell’infrastruttura esistente;
            \item Formazione sulle tecnologie adottate;
        \end{itemize}
        \item \textbf{Seconda Settimana - Sottotitolo (XX ore)} 
        \begin{itemize}
            \item ;
        \end{itemize}
        \item \textbf{Terza Settimana - Sottotitolo (XX ore)} 
        \begin{itemize}
            \item ;
        \end{itemize}
        \item \textbf{Quarta Settimana - Sottotitolo (XX ore)} 
        \begin{itemize}
            \item ;
        \end{itemize}
        \item \textbf{Quinta Settimana - Sottotitolo (XX ore)} 
        \begin{itemize}
            \item ;
        \end{itemize}
        \item \textbf{Sesta Settimana - Sottotitolo (XX ore)} 
        \begin{itemize}
            \item ;
        \end{itemize}
        \item \textbf{Settima Settimana - Sottotitolo (XX ore)} 
        \begin{itemize}
            \item ;
        \end{itemize}
        \item \textbf{Ottava Settimana - Conclusione (XX ore)} 
        \begin{itemize}
            \item ;
        \end{itemize}
    \end{itemize}
}

% Indicare il totale complessivo (deve essere compreso tra le 300 e le 320 ore)
\newcommand{\totaleOre}{}

\newcommand{\obiettiviObbligatori}{
	 \item \underline{\textit{min01}}: Analisi e redazione documentazione su IPFS privato;
	 \item \underline{\textit{min02}}: Sviluppo test IPFS privato;
	 \item \underline{\textit{min03}}: Analisi e redazione documentazione per integrazione con blockchain sviluppata dall'azienda;
	 	 \item \underline{\textit{min04}}: Interfaccia di login e inserimento metadati su hash;
	 
}

\newcommand{\obiettiviDesiderabili}{
	 \item \underline{\textit{max01}}: Sviluppo caso d'uso (tecnologia immatura);
	 \item \underline{\textit{max02}}: Effettiva integrazione con blockchain aziendale (tecnologia immatura);
	 \item \underline{\textit{max02}}: Upload e notorizzazione su blockchain tramite UI;
}

\newcommand{\obiettiviFacoltativi}{
	 \item \underline{\textit{for01}}: Comprensione dei concetti e delle tematiche legate alla blockchain, dei possibili campi di utilizzo, delle ragioni che ne identificano il suo potenziale dirompente e delle peculiarità di questa tecnologia rispetto ad altre soluzioni;
	 \item \underline{\textit{for02}}: Comprendere il funzionamento di IPFS;
	 \item \underline{\textit{for03}}: Analizzare pro e contro della versione privata di IPFS (supporto, maturità, ecc)
;
	 \item \underline{\textit{for04}}: Comprendere come realizzare un’interfaccia grafica efficace per le attività richieste.
}